\documentclass{article}
\usepackage{graphicx}
\usepackage{pgffor}
\usepackage{pdfpages}
\usepackage{geometry}
\geometry{margin=0pt} % quita márgenes
\usepackage{float}


\begin{document}

\foreach \i in {1,...,35}{%
    \clearpage
    \begingroup
      \edef\FileName{Imatges/d\i.png}%
      \begin{figure}[H]
        \centering
        \includegraphics[width=0.9\linewidth]{\FileName}
      \end{figure}
    \endgroup
}

\paragraph{{Final data description}}
The iFood dataset includes 2,031 customers across nine variables, showing no missing data and excellent quality. Customers are predominantly middle-aged and well-educated, with a wide but reasonable income range (mean €52,844). Family structures are typically small, and complaints are extremely rare (0.98\%). Income relates positively to education and age but negatively to having children, while higher-income, recently active customers respond more to marketing campaigns. Overall, the data are well-behaved, balanced across variables, and suitable for multivariate modeling, with only one extreme income outlier and an imbalanced “Complain” variable as minor considerations.

\end{document}
