\section{Association Rules}
\subsection{Objectives}
Association Rules are an unsupervised data mining method designed to discover interesting relationships, frequent patterns, or correlations among sets of items in large databases. 

Their most well-known use is market basket analysis (Ex: \textit{``If a customer buys A and B, is it likely that they will buy C?''}). However, we intend to use them to identify the customer profiles most likely to accept the latest marketing campaign.

To perform our association rule analysis, we have decided to use the following categorical variables from our enriched database that have the most influence on our results.

\begin{enumerate}
\item \textbf{Response}: Represents the acceptance of the last campaign. The purpose of the analysis is to find which combinations of attributes increase the probability of success ($\texttt{Response} = \texttt{Yes}$).

\item \textbf{IncomeSegment}: This variable is the principal interpreter of PC1 (Value). It differentiates clients based on their economic capacity (Low, Medium, High). This variable is quite important to consider as it is the most discriminating factor in the dataset.

\item \textbf{PreferredChannel}: In the PCA, we observed that the channel defines the behavior; that is, the "Discount/Web" client is very different from the traditional "Store/Catalog" client. For this reason, we can use it to determine where to launch the campaign. 

\item \textbf{PreferredProductCategory}: This allows linking the campaign's success to specific preferences, such as meat or wine. 

\item \textbf{MaritalSts}: This variable provides the social context. Although many profiles were similar, we were able to find, for instance, that the "\textit{Widow}" status was unexpectedly associated with high-income and traditional clients. 

\item \textbf{Education}: It acts as an indirect indicator of income level and sophistication. Previously, we were able to observe in the first deliverable that the "Basic" level was strongly linked to the low-income segment. This can help us filter who is not going to buy.
\end{enumerate}

Apart from these, we have decided to discretize the following variables to be able to consider them:

\begin{enumerate}
\item \textbf{Age}: This variable represents PC3 (Maturity). By discretizing it, we can detect whether the campaign works better across specific generational segments. To discretize it, we have decided to separate the ages into three segments:

\begin{itemize}   
    \item \textbf{Young (<= 35 years):} Customers in the early stages of their career/professional life.
    \item \textbf{Adult (36 - 55 years):} Customers with consolidated income, often with higher purchasing power.
    \item \textbf{Senior (55 < years):} Customers who are approaching or are in the retirement phase.
\end{itemize}


\item \textbf{WineSegment}: Given that wine is the star product, we can use this variable to distinguish between those who spend 'little' and those who spend 'a lot.' In short, it is vital for finding profitability niches. 

To discretize it, we used the WineExp variable and discretized it by quantiles (terciles) to obtain three segments of similar size.

\item \textbf{Teenhome}: This was the most influential variable of PC2. Its presence radically changes the way customers shop because it forces a search for deals. Since the number of teenagers at home is never more than five in the best-case scenario, we can discretize it easily into groups with the same number of teenagers.

\item \textbf{HasChildren}: In our initial analysis, we were able to confirm that having young children reduces the budget for luxuries (such as wine or gold). It is fundamental to explain why certain customers, even if they want to, do not spend much. To use this variable, we have converted the response into the binary "Yes" or "No".
\end{enumerate}
In summary, this selection of variables covers the three dimensions found in the PCA:

\begin{enumerate}
    \item \textbf{Value Dimension:} Covered by \verb|IncomeSegment| and \verb|WineSegment|.
    \item \textbf{Life Cycle Dimension:} Covered by \verb|Teenhome|, \verb|HasChildren| and \verb|PreferredChannel|.
    \item \textbf{Demographic Dimension:} Covered by \verb|Age|, \verb|Education| and \verb|MaritalSts|.
\end{enumerate}
By using these specific variables, we ensure that the generated rules have a business explanation that is coherent with the previous analysis.

\subsection{Choosing algorithm: Apriori}
Before obtaining the association rules, we debated which algorithm to use to find them.

Given that customers who accept a campaign are a minority in our database, our goal is to find small niches of customers with the same characteristics who are likely to accept the latest marketing campaign. For this reason, we have decided to use the Apriori algorithm, as it is designed to generate and evaluate association rules based on directional metrics such as Confidence and Lift.

Conversely, we discard the use of the Eclat algorithm, as it specializes in the detection of frequent itemsets through vertical intersections, which does not align with our interests.

\subsection{Evaluation}
Once the algorithm has been chosen, we need to adjust the support and confidence to find the rules that interest us most.

Since we are looking for specific groups of customers, we have set the Support to 1\% (20 people aprox.).

Furthermore, given that the percentage of people who accept the campaign (Response = Yes) is 15\%, any Confidence level superior to 15\% would already be better than the average. For this reason, we decided to set the Confidence at 30\%, because it would double the current efficiency of the campaign.

Initially, we ran those lines in R.
\begin{lstlisting}
rulesApriori <- apriori(db_transaciones, 
                parameter = list(support = 0.01, 
                                confidence = 0.25, 
                                minlen = 2))
summary(rulesApriori)
\end{lstlisting}

The resulting output was:

\includegraphics[width=1.0\textwidth]{Images/AprioriOutput0.png}

From this, we observed that a total of 81,476 rules were found. This quantity of rules is far too complex to analyze correctly. Furthermore, the mean lift (1.3979) indicates that the average rule barely achieves a result better than random.

However, it is worth noting that the maximum lift (6.7991) indicates that within those 80,000 rules, there are very valuable 'gold nuggets,' but they are hidden among thousands of irrelevant rules.

Given such an excess of noise and a complexity difficult to manage, we decided to target directly those who have accepted the campaign with the following line of code:

\begin{lstlisting}
rulesApriori <- apriori(dtrans, 
                parameter = list(supp = 0.01,
                                 conf = 0.3,
                                 minlen = 2),
                appearance = list(rhs = "Response=Yes", default="lhs"))
summary(rulesApriori)
\end{lstlisting}

In this way, we obtain these better results:

\includegraphics[width=1.0\textwidth]{Images/AprioriOutput1.png}

Following this second analysis, we have reduced the rules from 81,476 to 313, which is broad enough to find several distinct customer profiles.

Using direct target to positive results, we see that the average lift (2.615) improves the initial result, as the rules are more than twice as efficient as the average.

Furthermore, the maximum lift (4.301), despite being lower than the previous one, is still a very good result.

If we look at the confidence value, we see that it ranges from 30\% to 65\%. Although this may not seem very high at first glance, let's remember that the success rate of the Response variable is 15\%. Therefore, in the worst-case scenario, we are finding niches that double that percentage.

Finally, the support and count can help us determine the size of our niche. Oscillating between 0.01 and 0.07, we see that the customer group that interests us ranges from 21 to 139 people.

Visually, we can see these rules through these plots:

\begin{figure}[htbp]
    \centering
    \includegraphics[width=0.48\textwidth]{Images/LiftXSuport1.png}
    \hfill
    \includegraphics[width=0.48\textwidth]{Images/LiftXSuport2.png}
    \caption{Scatter plots of Lift vs Support for association rules}
\end{figure}

From these results, we can clearly observe that both the lift and the confidence align with the expected outcomes. We see that most associations with high lift and high confidence have a very small support, which makes sense since these types of customers form a very small group of people.

Furthermore, we notice that the most abundant orders (rule lengths) have between 4 and 6 antecedents. This means that a positive instance of Response is not exclusively related to 2 or 3 simple rules, but rather is formed by a more complex group of conditions.

To find these specific rules, we decided to separate the top 10 with this code snippet:

\begin{lstlisting}
rulesFiltered <- head(sort(rulesApriori, by = "lift"), 10)
plot(rulesFiltered, method = "paracoord", measure = "confidence", 
     control = list(reorder = TRUE))
\end{lstlisting}

After executing it, we obtain the following parallel coordinates plot, which serves to break down the complexity of the rules and allows us to visualize which 'variables' are combined and in what order to achieve the final result (Response=Yes).

\includegraphics[width=1.0\textwidth]{Images/ParallelCoordinates.png}

Analyzing the trajectories, we see that this graph reaffirms that the success of the campaign does not depend on a single isolated variable, but on the alignment of multiple factors.

The probability of purchase (rhs) only spikes when these 4-5 factors are aligned. This confirms the need for precise, non-generalist segmentation.

The conjunction of these rules reveals the characteristics of the perfect customer profile we are looking for.

\begin{enumerate} 
\item \textbf{Value Dimension:} The perfect customer has high economic capacity, explicitly defined by \verb|IncomeSegment = High| and a clear predisposition to spending with \verb|WineSegment = Wine_High|. 

\item \textbf{Life Cycle Dimension:} The perfect customer is notable for the absence of dependent children, marked by \verb|Teenhome = 0| and \verb|HasChildren = No|, which frees up their disposable income.

\item \textbf{Demographic Dimension:} The perfect customer is single and has a high level of education, indicated by \verb|Education = Graduation| and \verb|MaritalSts = Single|.
\end{enumerate}

Finally, we take a look at the top 10 rules with the best lift and the graph to confirm this.

\begin{table}[htbp]
    \centering
    \caption{Top 10 Additional Association Rules for \textbf{Response = Yes}}
    \label{tab:reglas_apriori_grid_2}
    \footnotesize
    
    \renewcommand{\arraystretch}{1.3}
    
    \begin{tabular}{|c|p{7cm}|c|c|c|c|}
        \hline
        \textbf{\#} & \textbf{Antecedent (LHS)} & \textbf{Supp.} & \textbf{Conf.} & \textbf{Lift} & \textbf{Count} \\
        \hline
        
        1 & \{Education=Graduation, MaritalSts=Single, Teenhome=0, IncomeSegment=High, WineSegment=Wine\_High\} 
          & 0.013 & 0.659 & 4.30 & 27 \\ \hline
          
        2 & \{Education=Graduation, MaritalSts=Single, HasChildren=No, IncomeSegment=High, WineSegment=Wine\_High\} 
          & 0.013 & 0.650 & 4.24 & 26 \\ \hline
          
        3 & \{Education=Graduation, MaritalSts=Single, Teenhome=0, HasChildren=No, IncomeSegment=High, WineSegment=Wine\_High\} 
          & 0.013 & 0.650 & 4.24 & 26 \\ \hline
          
        4 & \{PreferredChannel=WebPurc, HasChildren=No, IncomeSegment=High, WineSegment=Wine\_High\} 
          & 0.011 & 0.639 & 4.17 & 23 \\ \hline
          
        5 & \{Teenhome=0, PreferredChannel=WebPurc, HasChildren=No, IncomeSegment=High, WineSegment=Wine\_High\} 
          & 0.011 & 0.639 & 4.17 & 23 \\ \hline
          
        6 & \{Education=Graduation, MaritalSts=Single, Teenhome=0, WineSegment=Wine\_High\} 
          & 0.014 & 0.630 & 4.12 & 29 \\ \hline
          
        7 & \{PreferredChannel=WebPurc, HasChildren=No, IncomeSegment=High\} 
          & 0.012 & 0.625 & 4.08 & 25 \\ \hline
          
        8 & \{Teenhome=0, PreferredChannel=WebPurc, HasChildren=No, IncomeSegment=High\} 
          & 0.012 & 0.625 & 4.08 & 25 \\ \hline
          
        9 & \{Education=Graduation, MaritalSts=Single, HasChildren=No, WineSegment=Wine\_High\} 
          & 0.014 & 0.622 & 4.06 & 28 \\ \hline
          
        10 & \{Education=Graduation, MaritalSts=Single, Teenhome=0, HasChildren=No, WineSegment=Wine\_High\} 
           & 0.014 & 0.622 & 4.06 & 28 \\ \hline
    \end{tabular}
\end{table}

\begin{figure}[htbp]
    \centering
    \includegraphics[width=1.0\textwidth]{Images/Grafo.png}
    \caption{Graph visualization of top association rules}
\end{figure}

In this way, we have been able to identify which association rules allow us to improve our future marketing campaigns. To explain this clearly, we have divided them into two categories based on their importance.

The identified rules demonstrate a fourfold increase in campaign acceptance rates compared to the baseline.

\begin{enumerate}
\item \textbf{Common variables:} Regardless of the profile, there are two non-negotiable conditions that appear in all top rules.

First, the customer must not have children \textbf{(Teenhome = 0 or HasChildren = No)}, meaning that the campaign only works if the customer does not have the economic pressure or time constraints that children/teenagers impose.

Second, the customer must possess high economic value. Whether through \textbf{IncomeSegment=High or WineSegment=Wine\_High}, the customer must have spending capacity.

\item \textbf{Archetypes:} From here the rules diverge, which allows us to identify two customer archetypes:
\begin{itemize}   
        \item \textbf{Single with High Discretionary Spending Potential:}
        This archetype is reflected in rules 1, 2, 3, 6, 9, and 10 and is characterized by being single, having a high level of education, and demonstrating a propensity to spend money on wine. For this profile, spending behavior (Wine\_High) is more important than theoretical income. 
        
        According to our results, a single graduate who spends heavily on wine is a "top" customer, even if their salary is not necessarily classified as "High". Their behavior validates their value.
        
        \item \textbf{The Affluent Digital Buyer:}
        This archetype is reflected in rules 4, 5, 7, and 8 and is characterized by the absence of children, high income level, and a preference for purchasing through web orders. Unlike the previous archetype, it does not matter if the customer is single or a graduate.
        
        In conclusion, if the customer has high income, no children, and prefers to shop online, they will accept the campaign even if they are married or have a different education level. The digital channel is their enabler.
\end{itemize}
\end{enumerate}

\subsection{Recommended Strategy Conclusions}
Based on our analysis, our recommendation is not to create a single campaign, but rather a bifurcated strategy:

\begin{enumerate}
    \item \textbf{Strategy 1:}
    This strategy will target customers from the first archetype. The goal is to appeal to customers with the highest wine spending with a message that conveys exclusivity, personal reward, and a gourmet lifestyle. 
    
    \item \textbf{Strategy 2:}
    This strategy will target customers from the second archetype. To do this, we will focus on their preferred channel (web), where we will offer easy recommendations, direct access, and exclusive products just one click away.
\end{enumerate}

By focusing on these valuable customers, we can increase the percentage of people who accept a future campaign.
